\section{IEEE 802.11ah Parameters Selection \label{sec:parameters_selection}}


% However, it assumes homogeneous networks, i.e., all stations uses same \gls{mcs} and packet size, and stations are evenly divided into RAW groups. Moreover, some network-related parameters, such as beacon interval, channel width, station transmission queue size are not considered as fixed. 


% The 802.11ah standard, however, does not specify how to configure the actual \gls{raw} grouping parameters. Previous research ~\cite{WoWMoM2016} has shown that the optimal \gls{raw} configuration depends on network-related parameters, such as the number of stations, traffic patterns, and network load~\cite{WoWMoM2016}.

In comparison to previous research \cite{WoWMoM2016}, a more broad range of parameters are evaluated for \gls{raw} performance, the main difference are as follows. First, more realistic network scenarios are considered, stations can have different \gls{mcs}, packet size, transmission interval and transmission queue size, the stations can be parsley/densely located with same/different distance to \gls{ap}. Second, our experimentation evaluates a more diverse \gls{raw} grouping method, instead of dividing airtime and stations evenly among groups, stations assignment and \gls{raw} group duration varies based on different metrics. Table \ref{tab:ns3 parameters} lists the parameters and their values used in experimentation. In total 17 parameters are chosen, 12 of them are parameters for network, the others are \gls{raw} related parameters.

The channel width of IEEE 802.11ah ranges from 1 to  16 MHz, with only 1 and 2 MHz support mandatory. As we aim for \gls{iot} application, 1 and 2 MHz are evaluated in this experiment. The parameters $p_\textit{1M}$ indicates the ratio of stations using 1 MHz channel width. $sta_\textit{dis}$ specifies whether stations are randomly located around \gls{ap} or stations are distributed along a circle. $d_\textit{max}$ defines the maximal distance between stations and \gls{ap}, in case of \textit{circular} station distribution, all stations use the maximal distance as their distance to the \gls{ap}. As the transmission range of IEEE 802.11ah is up to 1000 meters, we choose 21 values for this parameter, ranging from 5 to 1000 meters. As the stations can adapt \gls{mcs} based on their distance to the \gls{ap} or some rate control algorithms \cite{rca2006}, this parameter also implies a variety of \gls{mcs} for stations. Three parameters $s_\textit{dis}$, $s_\textit{min}$ and $s_\textit{max}$ jointly determines the packet size of stations. The value \textit{uniform} for parameters $s_\textit{dis}$ means the packet size follows uniform distribution on the interval [$s_\textit{min}$  $s_\textit{max}$]. While packet size follows gaussian distribution with a mean of ($s_\textit{min} + s_\textit{max}) / 2 $ \textit{gaussian} and a a standard deviation of 1.0. As the \gls{iot} scenarios are quite diverse, and they requires a variety of traffic types, similar to packet size distribution, we use parameters $i_\textit{dis}$, $i_\textit{min}$  and $i_\textit{max}$ to generate different traffic types. The value \textit{uniform static} and \textit{gaussian static} for parameters $i_\textit{dis}$ both indicate that stations send packets with a fixed time interval, while transmission interval follows uniform distribution on the interval [$i_\textit{min}$  $i_\textit{max}$] for \textit{uniform static}, and follows  gaussian distribution with a mean of ($i_\textit{min} + i_\textit{max})/ 2 $ \textit{gaussian} and a a standard deviation of 1.0 for \textit{gaussian static}. On the contrary, \textit{poisson variable} allows stations to transmit packets at a poisson rate of ($i_\textit{min} + i_\textit{max}) / 2 $. Since transmission queue length also has an impact on the network performance \cite{Duffy2007}, we investigate different transmission queue lengths $l_q$, ranging from 1 to infinite. 5 different beacon interval $\mathcal{B}$ are evaluated to assess the impact of beacon frame overhead on network performance. The station number varies from 25 to 2000 stations, with steps of 15, 50 and 250. %The three different steps are designed in order to cope with \gls{raw} parameters, explanation will be given in next paragraph.

As indicated by parameter $g_r$, Stations are split into $n_r$ distinct clusters (\gls{raw} group) based on a variety of distance metrics, i.e., \textit{MCS}, \textit{packet size}, \textit{TX time}, \textit{random} and \textit{distance}. Among them, \textit{TX time} means packet transmission time, and \textit{random} allows station to be randomly clustered. \gls{raw} groups ranging from 1 to 40 are evaluated with steps of 1, 2 and 5. %The three different steps are designed in such way in order to cover
The \gls{raw} group duration is determined by parameter $d_r$, it assigns airtime among \gls{raw} groups in proportion to station number, packet transmission number and average packet transmission time when $d_r$ is set to \textit{stations} , \textit{packets} and \textit{TX time} respectively. When $d_r$ is \textit{uniform}, airtime is divided into \gls{raw} groups equally. The experiment support (non)cross slot boundary by setting $c_r$ to \textit{off}(\textit{on}). The number of RAW slots per group is defined as $\textit{sl}_r \times 64$, where $64$ is maximal number of \gls{raw} slots supported by IEEE 802.11ah.








% revious research ~\cite{WoWMoM2016} has shown that the optimal \gls{raw} configuration depends on network-related parameters, such as the number of stations, traffic patterns, and network load~\cite{WoWMoM2016}

% Third, to our
% knowledge, we are the first to consider a broad range of RAW
% parameters (i.e., group count, duration and station division
% among groups), network parameters (i.e., number of stations,
% network load, traffic patterns) and evaluation metrics (i.e.,
% throughput, latency and energy-efficiency) and to consider
% their respective influence on each other

\begin{table*}[t]
\centering
\renewcommand{\arraystretch}{1.2}
%\tiny
\caption{Simulation parameters used for experiments\label{tab:ns3 parameters}}
\begin{tabular}{lll}
\hline
\textbf{Parameters}  & \textbf{Description}            		 & \textbf{Value}  \\
\hline
 $\mathcal{B}$ & Beacon interval (ms)            		        & 102.4, 204.8, 409.6, 1024, 2048  \\
 $\mathcal{N}$ & Station number              		        & 25, 50, 75, 100, 150, 200, …, 500, 750, …, 1750, 2000    \\
 $p_\textit{1M}$  & Bitrate probably 1 MHz                      & 0, 0.1, 0.2, …, 0.9, 1      \\  
 $sta_\textit{dis}$ & Station distribution         				& random, circular \\
 $d_\textit{max}$ & Maximum station distance from the AP (m)       & 5, 50, 100, 150, …, 950, 1000 \\
 $s_\textit{dis}$ & Packet size distribution          & uniform, gaussian  \\
 $s_\textit{min}$ & Minimum packet size (bytes)       & 16, 32, …, 240, 256   \\
 $s_\textit{max}$ & Maximum packet size (bytes)       & MIN, …, 240, 256  \\
 $i_\textit{dis}$ & Packet arrival rate distribution  & uniform static, gaussian static, poisson variable\\
 $i_\textit{min}$ & Minimum packet arrival interval   & 1, 10, 30, 60, 600, 1200, 3600   \\
 $i_\textit{max}$ & Maximum packet arrival interval   & MIN, …, 3600  \\
 $l_q$ & Station transmission queue size (packets)   & 1, 10, 20, 50, 100, infinite   \\
 $g_r$ & RAW grouping strategy             & MCS, packet size, TX time, random, distance \\
 $n_r$ & Number of RAW groups              & 1, 2, 3, …, 9, 10, 12, 14, …, 20, 25, .., 40  \\
 $d_r$ & RAW group duration                & uniform, stations, packets, TX time  \\
 $c_r$ & Cross slot boundary               & on, off \\
 $\textit{sl}_r$ & Number of RAW slots per group     & 0, 0.01, 0.025, 0.05, 0.075, 0.1, 0.25, 0.5, 0.75, 1 \\

\hline
\end{tabular}
\end{table*}

