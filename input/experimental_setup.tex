\section{Experimental setup \label{sec:setup}}

%The simulator, experimental set up, how some of the parameters were implemented

As there is no IEEE 802.11ah hardware available, All the experiments are performed using the extended version of IEEE 802.11ah ns-3 module \cite{WNS32018}. The implementation builds upon existing 802.11ah implementations in ns-3 \cite{wns32016}, it support a physical layer model for sub-1GHz radio communications and a subset of the new MAC layer features of the standard, such as \gls{raw} and its dynamic configuration interface, energy consumption model and TIM segmentation. The experiment use the default PHY parameters as shown in Table \ref{tab:phy parameters}.

Some of the parameters listed in table \ref{tab:ns3 parameters} can be directly configured in the simulator, including network-related parameters $\mathcal{B}$, $\mathcal{N}$, $l_q$,  $p_\textit{1M}$, $sta_\textit{dis}$ and $d_\textit{max}$, \gls{raw}-related parameters $c_r$, $\textit{sl}_r$ and $n_r$. Based on parameters  $s_\textit{dis}$, $s_\textit{min}$ and $s_\textit{max}$, a \textcolor{red}{new}  function is created to generate a packet size for each station. Similarly, another \textcolor{red}{new} function is created to generate a transmission interval for each station according to parameters $i_\textit{dis}$, $i_\textit{min}$, and $i_\textit{max}$.\textcolor{red}{New function} is created to cluster stations and determines the \gls{raw} group duration based on the $g_r$ and $d_r$ parameters, then configure the rps element via the \gls{raw} configuration interface.

As mentioned in section \ref{sec:parameters_selection}, the stations can adapt \gls{mcs} based on their distance to the \gls{ap} or some rate control algorithms \cite{rca2006}, this parameter also implies a variety of \gls{mcs} for stations. In this experiment, we use distance as the single factor for determining the \gls{mcs}.

  




% \begin{table*}[t]
% \centering
% \renewcommand{\arraystretch}{1.2}
% \caption{Simulation parameters used for experiments\label{ns3parameters}}
% \begin{tabular}{lll}
% \hline
% \textbf{Parameters}  & \textbf{Description}            		 & \textbf{Value}  \\
% \hline


%  $g_r$ & RAW grouping strategy             & MCS, packet size, TX time, random, distance \\
%  $d_r$ & RAW group duration                & uniform, stations, packets, TX time  \\
% \hline
% \end{tabular}
% \end{table*}

\begin{table}[t]
\centering
\renewcommand{\arraystretch}{1.2}
%\tiny
\caption{PHY parameters \label{tab:phy parameters}}
\begin{tabular}{ll}
\hline
Frequency (Mhz)                & 868 \\
TX power (dBm)                 & 0    \\
TX/RX gain (dB)                & 0     \\
Noise Figure (dB)              & 6.8      \\  
Coding method                  & BCC \\
Propagation model              & Outdoor, \cite{globecom2017} \\ %, macro~\cite{Hazmi2012} \\
Error rate model               & YansErrorRate \\
\hline
\end{tabular}
\end{table}


% We consider the same IoT scenario as described in
% Section IV. The same default PHY and MAC layer parameters
% used as shown in Table I. We consider both homogeneous
% and heterogeneous scenarios. Homogeneous scenarios are
% used to validate the surrogate model and compare to ETAROA
% [8]. In heterogeneous scenarios, half of the stations
% use the high-thoughput (HT) settings and half of them use
% the low-throughput settings (LT) listed in Table I. The data
% transmission interval of each station is selected uniformly at
% random from the interval [1; 10] seconds.